\documentclass[12pt,letterpaper,leqno]{article}
\usepackage[latin1]{inputenc}
\usepackage[left=1in,right=1in,top=1in,bottom=1in]{geometry}
\usepackage{amsmath}
\usepackage{amsfonts}
\usepackage{amssymb}
\usepackage{url}
\usepackage{hyperref}
\usepackage[hang,flushmargin]{footmisc}
\hypersetup{
    pdftitle={Integral Table from http://integral-table.com},    % title
    pdfauthor={Shapiro},     % author
    pdfsubject={Table of Integrals},   % subject of the document
    pdfcreator={pdftex},   % creator of the document
    pdfproducer={Texmaker}, % producer of the document
    pdfkeywords={CSUN, Integrals, Table of Integrals, Math 280, Math 351, Differential Equations}, % list of keywords
    colorlinks=true,       % false: boxed links; true: colored links
    linkcolor=red,          % color of internal links
    citecolor=red,        % color of links to bibliography
    filecolor=red,      % color of file links
    urlcolor=red           % color of external links
}

\newcommand{\dx}{\hspace{2pt}dx}
\newcommand{\dd}[1]{\hspace{2pt}d#1}
\usepackage{multicol}

\begin{document}
\pagestyle{empty}

\begin{center}
\section*{Table of Basic Integrals\footnote{\copyright 2014. From  \url{http://integral-table.com}, last revised \today. This material 
is provided as is without warranty or representation about the accuracy, correctness or suitability of this material for any purpose. This work is licensed under the Creative Commons Attribution-Noncommercial-ShareAlike United States License. }
}
\end{center}

\begin{multicols}{2}

\begin{equation}
\int x^n \dx = \frac{1}{n+1}x^{n+1}, \hspace{1ex} n\neq-1
\end{equation}

\begin{equation}
\int \frac{1}{x}\dx = \ln |x|
\end{equation}

\begin{equation}
\int u \hspace{2pt} \dd{v} = uv - \int v du
\end{equation}



\begin{equation}
\int e^x \dx = e^x 
\end{equation}

\begin{equation}
\int a^x \dx = \frac{1}{\ln a} a^x
\end{equation}

\begin{equation}
\int \ln x \dx = x \ln x - x
\end{equation}


\begin{equation}
\int \sin x \dx = -\cos x
\end{equation}

\begin{equation}
\int \cos x \dx = \sin x
\end{equation}

\begin{equation}
\int \tan x \dx = \ln |\sec x| 
\end{equation}

\begin{equation}
\int \sec x \dx = \ln |\sec x + \tan x|
\end{equation}

\begin{equation}
\int \sec^2 x \dx = \tan x
\end{equation}

\begin{equation}
\int \sec x \tan x \dx = \sec x
\end{equation}

\begin{equation}
\int \frac{a}{a^2+x^2}\dx = \tan^{-1}\frac{x}{a}
\end{equation}

\begin{equation}
\int \frac{a}{a^2-x^2}\dx = \frac{1}{2}\ln\left|\frac{x+a}{x-a}\right|
\end{equation}

\begin{equation}
\int \frac{1}{\sqrt{a^2-x^2}} \dx = \sin^{-1} \frac{x}{a}
\end{equation}

\begin{equation}
\int \frac{a}{x \sqrt{x^2-a^2}} \dx = \sec^{-1} \frac{x}{a}
\end{equation}

\begin{align}
\int \frac{1}{\sqrt{x^2-a^2}} \dx &= \cosh^{-1} \frac{x}{a} \\&= \nonumber \ln (x+\sqrt{x^2-a^2})
\end{align}

\begin{align}
\int \frac{1}{\sqrt{x^2+a^2}} \dx &= \sinh^{-1} \frac{x}{a} \\&=\nonumber \ln (x+\sqrt{x^2+a^2})
\end{align}

\end{multicols}

\end{document}